\documentclass[12pt,paper=a4,fleqn,headsepline,headings=normal,cleardoublepage=current,titlepage,twoside,bibliography=totoc,listof=totoc,numbers=noenddot,parskip=half]{scrbook}  % double sided document

\usepackage{latexsym,amsfonts,amssymb,amsmath,ngerman}
\usepackage{bibgerm} % german bibliography standards
\usepackage[section]{placeins}
\usepackage{pdfpages} 
\usepackage{graphicx}
\usepackage[utf8]{inputenc} % implementation of german umlaute
\usepackage[T1]{fontenc}
\usepackage{lmodern}
\usepackage{makecell}
\usepackage{longtable}
\usepackage{array}
\usepackage{units}
\usepackage[format=hang]{caption}
\usepackage{morefloats}
\usepackage{multirow}
\usepackage{footnote} % Ermöglicht Fußnoten in gleitenden Umgebungen
\usepackage{listings}
%\usepackage[framed,numbered]{mcode}
\usepackage[
%plainpages=false,
%pdfpagelabels,
pdftex,
colorlinks=true,
linkcolor=blue,
citecolor=blue,
bookmarks=true,
bookmarksopen=true,
bookmarksopenlevel=2,
pagebackref=false,
bookmarksnumbered=true,
pdfstartpage=1,
pdfstartview=FitH,
pdfpagemode=UseOutlines,
]{hyperref}

% Commands for figures:
\DeclareGraphicsExtensions{.pdf,.jpg,.png} % Dateiendung für Grafiken bei Erstellung eines PDF-Dokuments
\graphicspath{{figs/}} % Pfad für die Bilder
\def\figurename{Abbildung }

% Definition of type of thesis Arbeit, degree,
% and cooperating company
\newcommand{\abschluss}{Master}

\newcommand{\studiengang}{Maschinenbau}

\newcommand{\doctitle}{Aufbau einer 12-Phasen PMSM mit unabhängiger Phasenbestromung}

\newcommand{\firma}{Privatprojekt}

\newcommand{\firmenadresse}{Regensburg}

%\newcommand{\betreuer}{Prof. Dr.-Ing. Michael Saller}

\newcommand{\beginndatum}{01.06.2017}

%\newcommand{\abgabedatum}{30.09.2014}

%\newcommand{\bearbeiteranrede}{Herrn}

\newcommand{\bearbeitername}{Florian Bodensteiner\\ Johannes Gürtler\\Christian Simon}

%\newcommand{\matrikelnummer}{2746481}

\hypersetup{
pdftitle=\doctitle,
pdfauthor=\bearbeitername,
%pdfsubject=\studiengang,
%Enter 5-7 keyword for your work/area of work:
pdfkeywords={Multiphase, PMSM}
}

% -----------------------------------
% --- Main document -----------------
% ----|||||||||||||------------------
% ----vvvvvvvvvvvvv------------------
% -----------------------------------


\begin{document}
\thispagestyle{empty}
\pagenumbering{Alph}
\hypertarget{titelseite}{}
\pdfbookmark[0]{Titelseite}{titelseite}


\vspace*{1cm}

\vfill

{
 \huge
\begin{center}
\textbf{\doctitle}
\end{center}


\vfill


\vfill

\begin{center}
\bearbeitername \\
\end{center}

\vfill

\begin{center}
Email: simon.christian86@gmail.com\\
\end{center}

\vfill

\begin{center}
\begin{tabular}{ll}
Beginn:& \beginndatum \\
\end{tabular}
\end{center}

\vfill
}
\cleardoubleemptypage

\cleardoubleemptypage

% -----------------------------------
% --- Erklärung ---------------------
% ----|||||||||||||------------------
% ----vvvvvvvvvvvvv------------------
% -----------------------------------

\thispagestyle{empty}
\pagenumbering{roman}
\setcounter{page}{3}
\hypertarget{erklaerung}{}%
\pdfbookmark[0]{Erklärung}{erklaerung}%

\vfill

\cleardoubleemptypage

% -----------------------------------
% --- Zusammenfassung und Abstract -
% ----|||||||||||||------------------
% ----vvvvvvvvvvvvv------------------
% -----------------------------------

\chapter*{Abstract}

\vfill

\vfill
\cleardoubleemptypage

% -----------------------------------
% ---- Alle Verzeichnisse -----------
% ----|||||||||||||------------------
% ----vvvvvvvvvvvvv------------------
% -----------------------------------

\setcounter{tocdepth}{2}

\hypertarget{Inhalt}{}%
\pdfbookmark[0]{Inhalt}{Inhalt}%{\contentsname}
\tableofcontents
\listoffigures
\listoftables

% --- Indizierung ------------------

%\addchap*{Indizierung und Symbole}

%\section*{Indizierung}

%\begin{longtable}{ll}
%	$f_s$ & Schaltfrequenz\\
%	$P_{DRV}$ & Treiberleistung\\
%	$V_{ISOL}$ & Isolationsspannung\\
%	$V_{DC}$ & Zwischenkreisspannung\\
%	$V_{GE}$ & Gate-Emitter-Spannung\\
%	$V_{CE}$ & Collector-Emitter-Spannung\\
%	$V_{CES}$ & Collector-Emitter-Sättigungsspannung\\
%	$V_{AC(eff)}$ & Isolationstestspannung Gatetreiber(50Hz/1s)\\
%	$Q_G$ & Gateladung\\
%	$I$ & Strom\\
%	$U$ & Spannung\\
%	$R_{ON}$ & Bahnwiderstand\\
%	$R_{G}$ & Gatewiderstand\\
%	$R_b$ & Blockingtime Widerstand\\
%	$T_b$ & Blockingtime\\
%	$h$ & Schrittweite\\
%\end{longtable} 

%\section*{Symbole}
%
%\begin{longtable}{ll}
%	$x$& Skalar\\
%\end{longtable}
\cleardoublepage

% --- Abkürzungen ------------------

\addchap*{Abkürzungen}
\begin{longtable}{ll}
	PWM & Pulsweitenmodulation\\
	MOSFET & MOS Field Effect Transistor\\
	VCC & Versorgungsspannung\\
	GND & Ground\\
	EMV & Elektromagnetische Verträglichkeit\\
	DSP & Digital Signal Processor\\
	CAN & Controller Area Network\\
	I/O & Input/Output\\
	DC & Gleichstrom\\
	AC & Wechselstrom\\
	PCB & Printed Circuit Board\\
	ESR & Equivalenter Serienwiderstand\\
	LDO & Low Drop Out\\
	GAIN & Verstärkung\\
	ADC & Analog/Digital Converter\\
	TVS & Transient Voltage Supressor (Diode)\\
\end{longtable}

\cleardoublepage

% -----------------------------------
% ---- Hauptteil --------------------
% ----|||||||||||||------------------
% ----vvvvvvvvvvvvv------------------
% -----------------------------------

\pagenumbering{arabic}
\chapter{Gesamtsystem}
\label{ch:SYS}

\section{Anforderungen}
\label{sec:SYS_anforderungen}

\section{Recherche}
\label{sec:SYS_recherche}

\section{Konzept}
\label{sec:SYS_konzept}	
\chapter{Mechanik}
\label{ch:MECH}

\section{Motor}
\label{sec:MECH_motor}

\section{Prüfstand}
\label{sec:MECH_pruefstand}


	
		
\chapter{Hardware}
\label{ch:HW}

\section{System Host}
\label{sec:HW_systemhost}

\section{Motorsteuerung}
\label{sec:HW_motorsteuerung}		

\section{Strommessung}
\label{sec:HW_strommessung}

\section{Encoder}
\label{sec:HW_encoder}

\section{Wifi Host}
\label{sec:HW_wifihost}

\section{Spannungsversorgung}
\label{sec:HW_spannungsversorgung}	
		
	
\chapter{Software}
\label{ch:SW}

%-----------------------------------------------------------%
%----------------- Section for System Host -----------------%
%-----------------------------------------------------------%

\section{System Host}
\label{sec:SW_systemhost}

\subsection{Protocol}
\label{subsec:SW_protocol}

\subsection{UART}
\label{subsec:SW_uart}

\subsection{CAN}
\label{subsec:SW_can}

\subsection{PWM}
\label{subsec:SW_pwm}

\subsection{ADC}
\label{subsec:SW_adc}

\subsection{SPI}
\label{subsec:SW_spi}

\subsection{I$ ^{2} $C}
\label{subsec:SW_i2c}

\subsection{TIMER}
\label{subsec:SW_timer}

\subsection{QEI}
\label{subsec:SW_qei}

	
%-----------------------------------------------------------%
%-------------- Section for Controls       -----------------%
%-----------------------------------------------------------%

\section{Steuerungssoftware}
\label{sec:SW_steuerungssoftware}

\subsection{Open Loop Control}
\label{subsec:SW_openloopcontrol}

\subsection{Closed Loop Control}
\label{subsec:SW_closedloopcontrol}
		

% ---- Ende des Hauptteils --------------------

\bibliographystyle{gerplain}
\bibliography{literatur}

%%	Anhang
%% ------------------------------------------------------------------------	
\clearpage						% Neue Seite erstellen, Alle Tabellen 
%%%%%%										% müssen vor dieser Seite eingefügt werden.
\pagenumbering{Roman}		% Anhang wird mit römischen Seitenzahlen versehen%%%%	
\appendix
%\input{appendix/Datenblätter}
\input{appendix/Schaltpläne}
%\input{appendix/Konstruktionszeichnungen}
%\input{appendix/Programmcode}
	
%========================================================================

\end{document}